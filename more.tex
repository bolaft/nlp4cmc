\begin{multicols}
\begin{rhetoricaltext} \footnotesize
\unit[1]{Hi ubuntu-ers,} 
%\\

\unit[2]{I'm in need of some directions.} 
\unit[3]{I want to run ubuntu for my desktop, \\
but the installer doesn't support root or boot on raid.  } \\ %voté de façon dispersée. }
\unit[5]{En rétorsion à ce résultat, les États-Unis, [\ldots]}  \\ %qui ont voté contre, ont annoncé...} 
%\unit[6]{Le Canada [\ldots] } %pourrait lui aussi  }
\unit[7]{De son côté, Israël [\ldots] } \\ %a fait part de...} 
\unit[8]{Cette adhésion intervient alors que l'État de Palestine tente d'être reconnu membre à part entière de l'ONU, 
et que le Conseil de sécurité des Nations unies étudie cette demande. } \\
\unit[9]{Devenir membre de l'UNESCO est %généralement 
perçu comme un signe positif dans une campagne d'adhésion aux Nations unies. } 
\source{fr.wikinews.org}%\footnote{\url{http://fr.wikinews.org/wiki/L\%27UNESCO_vote_l\%27adh\%C3\%A9sion_de_la_Palestine}}
\end{rhetoricaltext}
%\end{figure}
\\ Figure 1.1 : texte écrit en français
\newpage
%\begin{figure}
\begin{rhetoricaltext} \footnotesize
\unit[10]{UNESCO decided in a meeting of its General Council in Paris on Monday to admit Palestine as a full member.}\\
\unit[11]{107 votes approved entry, 14 opposed, and 52 abstained.}\\
\unit[12]{The move coincides with the Palestinians' bid for full membership in the United Nations. }\\
\unit[13]{The United Nations Security Council is currently reviewing the application. }\\
\unit[14]{Becoming a member of UNESCO is generally perceived as having a positive impact on all of its future bids.}
\source{en.wikinews.org}%\footnote{\url{http://en.wikinews.org/wiki/UNESCO_votes_in_favor_of_Palestine_membership}}
\end{rhetoricaltext}
%\end{figure}
\\Figure 1.2 :  texte écrit en anglais
\end{multicols}
\begin{center}
%\begin{figure}
%\begin{figure}
%{\hspace{30pt}\setlength{\compressionWidth}{160pt}
\multirel{Joint}{
	{\dirrel{}{{\rstsegment{\refr{1}}}}
		{Elaboration}{\rstsegment{\refr{2}}}
		{\hspace{3cm} Elaboration}{
			\multirel{Contrast}{
				{\rstsegment{\refr{3}\hspace{0,5cm}}   }
				%{\rstsegment{\refr{4}\hspace{0,5cm}}}
				{\rstsegment{\refr{5}\hspace{0,5cm}}}
				{\rstsegment{\refr{7}} }
			}
		}
	}
	{\rstsegment{\refr{8}\hspace{0,5cm}}}
	{\rstsegment{\refr{9}}}
}
\\Figure 1.3 :  structure rhétorique du texte de la figure 1.1
%}
%\raisebox{-2em}{}  \hspace{4cm}
%\caption{default}
%\label{fig:figure2}
%\end{figure}
%\end{figure}
\end{center}
%\end{twocolumn}
%\onecolumn
\caption{Extraits d'un article écrit en français mis en correspondance avec un article écrit en anglais et structure rhétorique du texte français}
\label{ExtraitsMisEnCorrespondance}
\end{figure}




%------------------------------------------------------------------------------
\subsection{A quoted message is an altered version of the original one}
\label{secQuotedMessageIsAltered}


The process of exchanging messages involves distinct software agents to transfer and deliver the messages to an user. % between the hosts. The 
When received, the emails are stored in dedicated servers until they are retrieved by the user.
%
The program used by users for retrieving and managing emails is called a mail user agent (MUA). 
The MUA reads and formats the message in email format to send it.

New content sent by a writer is in general read by the reader in its expected visual rendering. 
But this cannot be the case of previous message content for various reasons.



The email %header and body 
formats and encodings are specified by several RFC\footnote{The \textit{Request for Comments} are guidelines and protocols which result from various working groups involved in the Internet Standardization \url{https://tools.ietf.org/html}.}
%
The RFC 822 proposed in August 1982 specifies a Standard for the Format of ARPA Internet Text Messages which is the ancestor of the current message format. This RFC was obsoleted by the RFC 2822  proposed in April 2001 which defines the original Internet Message Format. This RFC was updated in October 2008 by the RFC 5322 and in turn obsoleted in March 2013 by the RFC 6854.
%
The format of the Internet message bodies has been sub-specified by several recommendations and updates; See as a main entrance the RFC 2045 from November 1996 or its more recent update, the RFC 5335 from September 2008. These specifications are called the Multipurpose Internet Mail Extensions (MIME).


% email header and body
% https://tools.ietf.org/html/rfc2822  April 2001
% http://tools.ietf.org/html/rfc5322 October 2008
% http://tools.ietf.org/html/rfc6854 March 2013
% Multipurpose Internet Mail Extensions   Format of Internet Message Bodies
% http://tools.ietf.org/html/rfc2045 November 1996
% https://tools.ietf.org/html/rfc5335 September 2008

As a matter of fact, messages in a discussion are handled by several distinct email software clients which are not always standards-compliant and totally compatible.
So each of them can integrate their own mechanisms 
for quoting the previous messages when including them as well as 
for wrapping the too long lines\footnote{Feature for limiting the line length by splitting them into multiple pieces of no more than 80 characters and make the text readable without any horizontal scrolling.}.
%
In addition, because of some user preferences or systems configurations, a thread may embed and interleave various posting styles.  

As a consequences a discussion is a mix of various transformation mechanisms which can be equated to a noisy canal. 
%
One of the transformation issue concerns the encoding and the decoding of the quoted parts which can be wrongly recoded at each exchange step since there is no more dedicated header information about it.

% Previous messages are not included as a verbatim copy of the original one.

% of 65-80 characters, 
% In text display, line wrap is the feature of continuing on a new line when a line is full, such that each line fits in the viewable window, allowing text to be read from top to bottom without any horizontal scrolling. 
%http://en.wikipedia.org/wiki/Word_wrap

% Programs used by users for retrieving, reading, and managing email are called mail user agents (MUAs).

% The process of replying to a message by including the original message does not ensure to provide a pure duplicate of the original. 





\begin{figure}
%\begin{multicols}{2}[]

%\begin{multicols}{1}[]
    %    \centering
\fbox {
    \parbox{\linewidth}{
        \begin{subfigure}[b]{0.9\textwidth}
\small
%\footnotesize
[Hi ubuntu-ers,]$^{S1}$\vspace{0.1cm}

[I'm in need of some directions.]$^{S2}$ [I want to run ubuntu for my desktop.]$^{S3}$\\ \ 
[But the installer doesn't support root or boot on raid.]$^{S4}$  \vspace{0.1cm}
%[I'm in need of some directions.]$^{S2}$ [I want to run ubuntu for my desktop,\\
%but the installer doesn't support root or boot on raid.]$^{S3}$  \vspace{0.1cm}

[It seems that there must be a way to convert to raid after the fact, as \\
I've seen for some other distros.]$^{S5}$\vspace{0.1cm}

[I've tried a couple of the other distro specific directions, and just %\
wind up with a kernel panic.]$^{S6}$  \vspace{0.1cm}

[Are there some ubuntu specific instructions or advice?]$^{S7}$  [Thanks.]$^{S8}$\vspace{0.1cm}

%-- \\ \ 
%[Regards,]$^{S8}$ \\ \ 
[John]$^{S9}$
%Rich
              %  \includegraphics[width=\textwidth]{gull}

                \caption{Source message.}
                \label{fig:exampleSource}
        \end{subfigure}%
}}
\vspace{0.2cm}
\\
       % ~ %add desired spacing between images, e. g. ~, \quad, \qquad etc.
          %(or a blank line to force the subfigure onto a new line)
\fbox {
    \parbox{\linewidth}{
        \begin{subfigure}[b]{0.9\textwidth}
\small
%\footnotesize
%[On Sun, 04 Dec 2005 15:45:13 -0600, John Doe\\
[On Sun, 04 Dec 2005 15:45:13 -0600, John Doe 
%On Sun, 05 Dec 2004 16:48:14 -0600, Rich Duzenbury
%<rduz-ubuntu@theduz.com> wrote:
<john@doe.com> wrote:]$^{R1}$\vspace{0.1cm}

%> [I'm in need of some directions.]$^{R2}$  [I want to run ubuntu for my desktop,\\
%> but the installer doesn't support root or boot on raid.]$^{R3}$\vspace{0.1cm}
> [I'm in need of some directions.]$^{R2}$  [I want to run ubuntu for my desktop.]$^{R3}$\\
> [But the installer doesn't support root or boot on raid.]$^{R4}$\vspace{0.1cm}

[It claims not to.]$^{R5}$ [It actually does work.]$^{R6}$\vspace{0.1cm}

[Create a separate /boot partition for the kernel and initrd to live in %\\
and then install to a / filesystem living on a RAID1 or RAID0 device %\\
that you create during the install.]$^{R7}$ [My ubuntu server in the office %\\
currently has 4 active disks doing RAID0+1 and running LVM on top with % \\
another disk about to go in in case 2 happen to fail in quick % \\
succession :-)]$^{R8}$\vspace{0.1cm}

> [It seems that there must be a way to convert to raid after the fact, as\\
> I've seen for some other distros.]$^{R9}$\vspace{0.1cm}

[It's also possible to convert after the fact using generic %\\
instructions and rebuilding the initrd to cope with the fact that it % \\
will need to run mdadm for you.]$^{R10}$\vspace{0.1cm}

%[Cheers,]$^{R10}$\vspace{0.1cm}

[Bob.]$^{R11}$ %\vspace{0.1cm}
%Jon.

%[P.S.]$^{R11}$ [This is a major sticking point for ubuntu and Debian acceptance\\
%on mission critical kit which should be addressed.]$^{R12}$ [It's not too tricky\\
%to boot and initrd off a separate boot partition.]$^{R13}$\vspace{0.1cm}

               % \includegraphics[width=\textwidth]{tiger}
                \caption{Reply message.}
                \label{fig:exampleReply}
        \end{subfigure}
}}
%\end{multicols}

%        ~ %add desired spacing between images, e. g. ~, \quad, \qquad etc.
%          %(or a blank line to force the subfigure onto a new line)
%        \begin{subfigure}[b]{0.4\textwidth}


%\begin{tabular}{*{2}{|l}|c|}
%\toprule
%\textbf{Source} & \textbf{Reply} & \textbf{Label}\\
%	\midrule
%S1  & & \\
%    & R1 & \\
%\textit{S2}  & > \textit{R2}& Start\\
%\textit{S3}  & > \textit{R3}& Inside\\
%\textit{S4}  & > \textit{R4}& End\\
%    & R5 & \\
%    & R6 & \\
%    & R7 & \\
%\textit{S5}  & > \textit{R8} & Start \& End\\
%    & R9 & \\
%%    & R10 & \\
%    & [...] & \\
%S7    &  & \\ \ 
%[...] \    &  & \\
%	\bottomrule
%\end{tabular}
%               % \includegraphics[width=\textwidth]{mouse}
%                \caption{A mouse}
%                \label{fig:mouse}
%        \end{subfigure}
        \caption{Example of a source message and its reply. Pseudo-sentences have been marked. The original layout has been slightly adapted to fit the document.}\label{fig:exampleSourCeReplyMessage}
%\end{multicols}

\end{figure}



\begin{figure}[ht!]\small\centering
\begin{tabular}{*{2}{|l}|c|}
\toprule
\textbf{Source} & \textbf{Reply} & \textbf{Label}\\
	\midrule
S1  & & \\
    & R1 & \\
\textit{S2}  & > \textit{R2}& \texttt{Start}\\
\textit{S3}  & > \textit{R3}& \texttt{Inside}\\
\textit{S4}  & > \textit{R4}& \texttt{End}\\
    & R5 & \\
    & R6 & \\
    & R7 & \\
    & R8 & \\
\textit{S5}  & > \textit{R9} & \texttt{Start\&End}\\
    & R10 & \\
%    & R11 & \\
    & [...] & \\
S7    &  & \\ \ 
[...] \    &  & \\
	\bottomrule
\end{tabular}

\caption{Alignment of sentences from the source and reply messages shown in Figure~\ref{fig:exampleSourCeReplyMessage} and labels inferred from the re-use of source message text. Labels are associated to source sentences.}
\label{fig:exampleSegmentationLabels}
\end{figure}


Figure~\ref{fig:exampleSourCeReplyMessage} shows an example of a source message\footnote{In the original message, $S3$ and $S4$ consist in one single sentence with a comma as a separator. We split it into distinct sentences in order to illustrate the \textit{Inside} label.} (Figure~\ref{fig:exampleSource}) and one of its reply (Figure~\ref{fig:exampleReply}).
Marked pseudo-sentences correspond to what an automatic process can produce. 
In this example, we can see that the reply message only re-uses four selected sentences from the source message; namely $S2$, $S3$, $S4$ and $S5$ which respectively correspond to the sentences  $R2$, $R3$, $R4$ and $R9$ in the reply message.
The author of the reply message deliberately discarded the remaining of the source message.

