%
% File konvens2014.tex
%
% This file is identical to the file konvens2012.tex 
% which was provided by  Jeremy Jancsary, <jeremy.jancsary@ofai.at>
% We have renamed it only so it is clear to authors who submit to 
% Konvens 2014 that they are using the right version.
%
% Contact: konvens2014@uni-hildesheim.de
%%
%% Based on the style files for ACL2008 by Joakim Nivre and Noah Smith
%% and that of ACL2010 by Jing-Shin Chang and Philipp Koehn
%% and that of ACL2012 by Maggie Lie and Michael White.

\documentclass[11pt,a4paper]{article}
\usepackage{konvens2014}
\usepackage[pass]{geometry}
\usepackage{times}
\usepackage{latexsym}
\usepackage{amsmath}
\usepackage{multirow}
\usepackage{url}

\usepackage{bm}
\usepackage{todonotes}
\usepackage{csquotes}
\usepackage{float}
\usepackage{array}

% subfigure
\usepackage{graphicx}
\usepackage{caption}
\usepackage{subcaption}

% Package pour les tableaux 
\usepackage{booktabs}

% plusieurs colonnes et modification de leur taille dans les tables
%\usepackage{multirow}

%\usepackage{setspace}
\usepackage{listings}
\lstset{ %
  basicstyle=\footnotesize
}


%\usepackage{rst}
\usepackage{multicol}

\usepackage{enumitem}




\DeclareMathOperator*{\argmax}{arg\,max}
\setlength\titlebox{6.5cm}    % Expanding the titlebox

\title{Exploiting the human computational effort dedicated to message reply formatting for training discursive email segmenters} %Act of speech boundary detection in email corpora


\author{First Author \\
  Affiliation / Address line 1 \\
  Affiliation / Address line 2 \\
  Affiliation / Address line 3 \\
  \url{email@domain} \\\And
  Second Author \\
  Affiliation / Address line 1 \\
  Affiliation / Address line 2 \\
  Affiliation / Address line 3 \\
  \url{email@domain} \\}

\date{}

\begin{document}
\maketitle
\begin{abstract}
In the context of multi-domain and multimodal online asynchronous discussion analysis, we propose an innovative strategy for manual annotation of dialog act (DA) segments. The process aims at supporting the analysis of messages in terms of DA.
Our objective is to train a sequence labelling system to detect the segment boundaries. %; i.e. to classify if a given sentence starts, ends or continues a segment.
The originality of the proposed approach is to avoid manually annotating the training data and instead exploit the human computational efforts dedicated to message reply formatting %; in particular 
when the writer replies to a message by inserting his response just after the quoted text appropriate to his intervention.
%
We describe the approach, propose a new electronic mail corpus %to the community 
and report the evaluation of segmentation models we built.
\end{abstract}


% -----------------------------------------------------------------------------
% INTRODUCTION

%------------------------------------------------------------------------------
\section{Introduction}
\label{sec:intro}

Automatic processing of online conversations (forum, emails) 
% about information or assistance needs 
is a highly important issue for the industrial and the scientific communities which care to improve existing question/answering systems, identify emotions or intentions in customer requests or reviews, detect messages containing requests for action or unsolved severe problems\ldots
%
%L'analyse automatique des conversations écrites en ligne asynchrones (e.g. forum, courriel) portant des demandes d'information (e.g. dépannage et assistance) constituent un fort enjeu scientifique et industriel : amélioration des systèmes de question/réponse actuels, détection de revues négatives de produits par des clients, identification de problèmes sévères non-résolus... 
%
%2010_gupta_W10-0202_email-Emotion-Detection-in-Email-Customer-Care
%2012_vinodkumar_email_Annotations-for-Power-Relations-on-Email-Threads
%2010_kim-W10-0507_modeling-social-and-content-dynamics-in-discussion-forum
%2012_qu_E12-1037_recommendation-prediction-of-user-interest-in-forum
%2013_chen_N13-1124_Identifying-Intention-Posts-in-Discussion-Forums
%2011_marin_W11-0706_detecting-forum-authority-claims
%2010_wang_P10-1027_recommendation-in-forum-blog
%2009_Stavrianou_asonam_Definition and Measures of an Opinion Model for Mining Forums
%2009_arora_N09-2010_Identifying Types of Claims in Online Customer Reviews.pdf
%
%2013_ott_Negative Deceptive Opinion Spam
%
%2011_qu_I11-1164_finding-problem-solving-threads-in-forum
%2010_lampert_naacl_Detecting Emails Containing Requests for Action
%
%Ce sujet propose de travailler sur la tâche de segmentation des courriels en français dans des discussions portant sur des demandes d'information. Ce travail vise à soutenir une analyse automatique des messages en DA.
%
%Les travaux existants se fondent généralement sur une pré-annotation des messages en actes du dialogue (\textit{DA}). Les \textit{DA} décrivent les fonctions communicatives portées dans chaque message (e.g. question, réponse, remerciement...) Austin FIXME. Kim FIXME proposent une liste d'actes spécifiques à la description des messages au sein de forums.
%Jusqu'à présent la majorité des travaux ont modélisé les discussions au niveau de leurs messages en les décrivant su

In most works, conversation interactions between the participants are modeled in terms of dialogue acts (DA) \cite{austin:1970}. The DAs describe the communicative function conveyed by each text utterance  (e.g.~question, answer, greeting,\ldots).
%
In this paper, we address the problem of rhetorically segmenting the new content parts of messages in online asynchronous discussions. 
The process aims at supporting the analysis of messages in terms of DA.
We pay special attention to the processing of electronic mails.

%La majorité des travaux modélisent les interactions entre intervenants en termes d'\textit{actes de dialogue} (DA) \cite{austin:1970}. Les DAs correspondent aux fonctions communicatives portées par chaque énoncé d'un texte (e.g. question, réponse, remerciement...). 
% La théorie des actes du dialogue \cite{austin:1970} propose de décrire les énoncés en termes des fonctions communicatives portées par chacun d'eux (e.g. question, réponse, remerciement...). 
%
The main trend in automatic DA recognition consists in using supervised learning algorithms to predict the DA conveyed by a sentence or a message \cite{joty:2013:sigdial}.
% L'approche dominante en reconnaissance automatique de DA consiste en l'usage d'algorithmes de classification supervisée \cite{joty:2013:sigdial} pour déterminer l'acte porté par une phrase ou un message. 
%
The hypothesized message segmentation results from the global analysis of these individual predictions over each sentence.
%Le découpage des messages découle alors des résultats d'analyse.
%
A first remark on this paradigm is that it is not realistic to use in the context of multi-domain and multimodal processing because it requires the building of training data which is a very substantial and time-consuming task.
%Une première critique de ce paradigme est que sa mise en oeuvre est laborieuse et coûteuse en temps d'annotation pour construire des données d'entraînement, ce qui n'est pas réaliste en contexte de traitement multi-domaine voire multi-modal.
%
A second remark is that the model does not have a fine-grained representation of the message structure or the relations between messages. Considering such characteristics could drastically improve the systems % for example
%(e.g.~
to allow to focus on specific text parts or to filter out less relevant ones. %). 
% Une seconde critique est que dans ce contexte, la connaissance de la structure des messages permettrait aux applications susvisées de se focaliser sur des parties spécifiques et filtrer les passages moins pertinents. 
Indeed, apart from the closing formula, a message may for example be made of several distinct information requests, the description of an unsuccessful procedure, the quote of third-party messages\ldots
% En effet, outre des formules de politesse, un message peut compter par exemple plusieurs expressions de besoin, décrire une procédure infructueuse et citer des portions de messages tiers. 
% faire référence à plusieurs contenus exprimés par différentes contributions tierces

So far, few works address the problem of message segmentation.
\cite{lampert:2009:emnlp} propose to segment emails in prototypical zones such as the author's contribution, quotes of original messages, the signature, the opening and closing formulas. 
In comparison, we focus on the segmentation of the author's contribution (what we call the new content part).
\cite{joty:2013:jair} identifies %topical segments of sentences which consist in 
clusters of topically related sentences through the multiple messages of a thread, without distinguishing email and forum messages. Apart from the topical aspect, our problem differs because we are only interested in the cohesion between sentences in nearby fragments and % but
 not on distant sentences.
% . In our approach we are more interested by rhetorically motivated segments.
%De par sa robustesse, cette dernière approche peut servir de base de comparaison.
%Or, peu de travaux se sont penchés sur la tâche de segmentation des courriels en tant que telle. Les rares travaux ne concernent pas le découpage en DA. \cite{lampert:2009:emnlp} s'intéressent à découper les courriels en des zones prototypiques telles que la contribution de l'auteur, les reprises de messages tiers, la signature et les formules d'appel et de clôture... \cite{joty:2013:jair} identifient des segments thématiques. De par sa robustesse, cette dernière approche peut servir de base de comparaison.


Despite the drawbacks mentioned above, a supervised approach remains %generally 
the most efficient and reliable method to solve classification problems in Natural Language Processing. 
% 
Our aim is to train a system to detect the segment boundaries, %; in other words 
i.e. to determine, through a classification approach, if a given sentence starts, ends or continues a segment.

\begin{figure}
\begin{minipage}{.63\textwidth}
%\begin{multicols}{2}[]
%\begin{multicols}{1}[]
    %    \centering
\fbox {
    \parbox{\linewidth}{
        \begin{subfigure}[b]{0.9\textwidth}
\small
%\footnotesize
  [Hi!]$^{S1}$\vspace{0.1cm}

[I got my ubuntu cds today and i'm really impressed.]$^{S2}$ [My\\ \ 
friends like them and my teachers too (i'm a student).]$^{S3}$ \\ \ 
[It's really funny to see, how people like ubuntu and start feeling geek\\ \ 
and blaming microsoft when they use it.]$^{S4}$\vspace{0.1cm}

[Unfortunately everyone wants an ubuntu cd, so can i download the cd\\ \ 
covers anywhere or an 'official document' which i can attach to\\ \ 
self-burned cds?]$^{S5}$\vspace{0.1cm}

[I searched the entire web site but found nothing.]$^{S6}$ [Thanks in advance.]$^{S7}$\vspace{0.1cm}

[John]$^{S8}$
                \caption{Original message.}
                \label{fig:exampleSource}
        \end{subfigure}%
}}
\vspace{0.2cm}
\\
       % ~ %add desired spacing between images, e. g. ~, \quad, \qquad etc.
          %(or a blank line to force the subfigure onto a new line)
\fbox {
    \parbox{\linewidth}{
        \begin{subfigure}[b]{0.9\textwidth}
\small
%\footnotesize
%[On Sun, 04 Dec 2005 15:45:13 -0600, John Doe\\
[On Sun, 04 Dec 2005, John Doe 
%On Sun, 05 Dec 2004 16:48:14 -0600, Rich Duzenbury
%<rduz-ubuntu@theduz.com> wrote:
<john@doe.com> wrote:]$^{R1}$\vspace{0.1cm}

> [I got my ubuntu cds today and i'm really impressed.]$^{R2}$ [My\\ \ 
> friends like them and my teachers too (i'm a student).]$^{R3}$\\ \ 
> [It's really funny to see, how people like ubuntu and start feeling geek\\ \ 
> and blaming microsoft when they use it.]$^{R4}$\vspace{0.1cm}

[Rock!]$^{R5}$\vspace{0.1cm}

> [Unfortunately everyone wants an ubuntu cd, so can i download the cd\\ \ 
> covers anywhere or an 'official document' which i can attach to\\ \ 
> self-burned cds?]$^{R6}$\vspace{0.1cm}

[We don't have any for the warty release, but we will have them for hoary, %\\ \ 
because quite a few people have asked. :-)]$^{R7}$\vspace{0.1cm}

[Bob.]$^{R8}$ %\vspace{0.1cm}
%Jon.
%[P.S.]$^{R11}$ [This is a major sticking point for ubuntu and Debian acceptance\\
%on mission critical kit which should be addressed.]$^{R12}$ [It's not too tricky\\
%to boot and initrd off a separate boot partition.]$^{R13}$\vspace{0.1cm}
               % \includegraphics[width=\textwidth]{tiger}
                \caption{Reply message.}
                \label{fig:exampleReply}
        \end{subfigure}
}}
%\end{multicols}
%        ~ %add desired spacing between images, e. g. ~, \quad, \qquad etc.
%          %(or a blank line to force the subfigure onto a new line)
%        \begin{subfigure}[b]{0.4\textwidth}
%\begin{tabular}{*{2}{|l}|c|}
%\toprule
%\textbf{Source} & \textbf{Reply} & \textbf{Label}\\
%	\midrule
%S1  & & \\
%    & R1 & \\
%\textit{S2}  & > \textit{R2}& Start\\
%\textit{S3}  & > \textit{R3}& Inside\\
%\textit{S4}  & > \textit{R4}& End\\
%    & R5 & \\
%    & R6 & \\
%    & R7 & \\
%\textit{S5}  & > \textit{R8} & Start \& End\\
%    & R9 & \\
%%    & R10 & \\
%    & [...] & \\
%S7    &  & \\ \ 
%[...] \    &  & \\
%	\bottomrule
%\end{tabular}
%               % \includegraphics[width=\textwidth]{mouse}
%                \caption{A mouse}
%                \label{fig:mouse}
%        \end{subfigure}
        \caption{An original message and its reply (\textit{ubuntu-users} email archive). Sentences have been tagged to facilitate the discussion. %The original layout has been slightly adapted to fit the document.
%Examples of a source message and its reply based on the \textit{ubuntu-users} email archive. Pseudo-sentences have been marked to facilitate the discussion.
        }\label{fig:exampleSourCeReplyMessage}
%\end{multicols}
\end{minipage}
\hfill
\begin{minipage}{.3\textwidth}
\small\centering
\begin{tabular}{*{2}{|l}|c|}
\toprule
\textbf{Original} & \textbf{Reply} & \textbf{Label}\\
	\midrule
S1  & & \\
    & R1 & \\
\textit{S2}  & > \textit{R2}& \texttt{Start}\\
\textit{S3}  & > \textit{R3}& \texttt{Inside}\\
\textit{S4}  & > \textit{R4}& \texttt{End}\\
    & R5 & \\
\textit{S5}  & > \textit{R6} & \texttt{Start\&End}\\
    & R7 & \\
%    & R11 & \\
    & [...] & \\
S6    &  & \\ \ 
[...] \    &  & \\
	\bottomrule
\end{tabular}
\caption{Alignment of the sentences from the original and reply messages shown in Figure~\ref{fig:exampleSourCeReplyMessage} and labels inferred from the re-use of the original message text. Labels are associated to the original sentences.}
\label{fig:exampleSegmentationLabels}
\end{minipage}
\end{figure}
%
The originality of the proposed approach is to avoid manually annotating the training data and instead to exploit the human computational efforts dedicated to a similar task in a different context of production~\cite{ahn:2006:computer}. 
% 
As recommended by the \textit{Netiquette}\footnote{Set of guidelines for Network Etiquette (\textit{Netiquette}) when using network communication or information services RFC1855. %\url{http://tools.ietf.org/html/rfc1855}.
}, when replying to a message (email or forum post), the writer should 
 ``summarize the original message at the top of its reply, %the message, 
or include (or "quote") just enough text of the original to give a context, in order to make sure readers understand when they start to read the response\footnote{It is true that some email software clients do not conform to the recommendations of Netiquette and that some online participants are less sensitive to arguments about posting style (many writers reply above  the original message).
%In top-posting style, the original message is included verbatim, with the reply above it. 
We assume that there are enough messages with inline replying available to build our training data.}.''  As a corollary, the writer should ``edit out all the irrelevant material.''
%
% include enough original material to be understood but no more.
Our idea is to use this effort, in particular when the writer replies to a message by inserting his response or comment just after the quoted text appropriate to his intervention. 
%
This posting style is called \textit{interleaved} or \textit{inline replying}.
%L'originalité de l'approche sera de ne pas construire manuellement le corpus d'entraînement mais d'exploiter le temps cognitif investi par d'autres pour une tâche supposée similaire \cite{ahn:2006:computer}. Cela lui vaut le qualificatif de \textit{paresseuse}.
%L'idée est d'exploiter le travail de découpage réalisé par un intervenant lorsqu'il répond à l'intérieur d'un message (\textit{inline replying}).
%
%
The so built segmentation model should be usable for any posting styles by applying it only on new content parts.
%
%
%~\ref{fig:exampleSourCeReplyMessage} 
Figure~\ref{fig:exampleSource} shows an example of an \textit{original} %\footnote{By \textit{source} message, we refer to a message which is replied to. By \textit{original} message, we loosely mean a source message. With this term, we want to point out more precisely the author's contributions.} 
message and, Figure~\ref{fig:exampleReply}, one of its \textit{reply}.
We can see that the reply message re-uses only four selected sentences from the original message; namely $S2$, $S3$, $S4$ and $S5$ which respectively correspond to sentences  $R2$, $R3$, $R4$ and $R6$ in the reply message.
The author of the reply message deliberately discarded the remaining of the original message.
%
The segment build up by sentences $S2$, $S3$, $S4$ and the one by the single sentence $S5$ can respectively be associated with two acts : a comment and a question.
% Some mail clients, in particular some configurations of Microsoft Outlook, 
%some email software clients are not standards-compliant and 
%some email software clients do not conform to the recommendations of Netiquette.
%Some online participants are less sensitive to arguments about posting style. % (
%Newer online participants, especially those with limited experience of Usenet, tend to be less sensitive to arguments about posting style.

%FIXME When a message is replied to in e-mail, Internet forums, or Usenet, the original can often be included, or "quoted", in a variety of different posting styles.
%
%The main options are {\em interleaved posting} (also called {\em inline replying}, in which the different parts of the reply follow the relevant parts of the original post), \textit{bottom-posting} (in which the reply follows the quote) or \textit{top-posting} (in which the reply precedes the quoted original message). 

%We use then a supervised approach to build models of the segmentation.


In Section~\ref{sec:buildingannotatedcorpusofsegmentedonlinemessageatnocost}, we explain our approach for building an annotated corpus of segmented online messages at no cost. 
In Section~\ref{sec:buildingTheSegmenter}, we describe the  system and the features we use to model the segmentation. 
After presenting our experimental framework in Section~\ref{sec:experimentalframework}, we report some evaluations for the segmentation task in Section~\ref{sec:experiments}. 
Finally, we discuss our approach in comparison to other works in Section~\ref{sec:relatedWork}.






% -----------------------------------------------------------------------------
% APPROACH


%------------------------------------------------------------------------------
\section{Building annotated corpora of segmented online discussions at no cost}
\label{sec:buildingannotatedcorpusofsegmentedonlinemessageatnocost}

We present the assumptions and the detailed steps of our approach.

%------------------------------------------------------------------------------
\subsection{Annotation scheme}
\label{sec:annotationscheme}


The basic idea is to interpret the operation performed by a discussion participant on the message he replies as an annotation operation. 
Assumptions about the kind of annotations depend on the operation that has been performed.
Deletion or re-use of the original text material can give hints about the relevance of the content: discarded material is probably less relevant than re-used one.

We assume that by replying inside a message and by only including some specific parts, the participant performs some cognitive operations to identify homogeneous self-contained text segments.
% in the original message.
%We assume that the quoted part consists in an homogeneous information unit 
%with respect to the context from which they are extracted. 
Consequently, we make some assumptions about the role played by the sentences in the original message information structure.
%For instance, we can assume that the first sentence of a quoted part conveys instructions for opening a new discourse segment while the last sentence carries instructions for ending a segment.
%Section~\ref{sec:annotationscheme} presents our annotation scheme in detail.
%
%We assume that a message can be split into %subsequent and
% consecutive discourse segments, each of them conveying its own dialogue act.
%We take the sentence as the elementary unit.
%Consequently, each 
A sentence in a segment plays one of the following roles: %\\
%\indent $\bullet$ 
\texttt{\footnotesize starting and ending} (\textit{SE}) a segment when there is only one sentence in the segment, %\\
%\indent $\bullet$ 
\texttt{\footnotesize starting} (\textit{S}) a segment if there are at least two sentences in the segment and it is the first one, %\\
%\indent $\bullet$
 \texttt{\footnotesize ending} (\textit{E}) a segment if there are at least two sentences in the segment and it is the last one, %\\
%\indent $\bullet$ 
\texttt{\footnotesize inside} (\textit{I}) a segment in any other cases.
%
%\begin{description}[itemsep=0mm]
%\itemsep0em 
%\item [ starting and ending] (\textit{SE}) a segment when there is only one sentence in the segment, 
%\item [ starting] (\textit{S}) a segment if there are at least two sentences in the segment and the sentence is the first one, 
%\item [ ending] (\textit{E}) a segment if there are at least two sentences in the segment and the sentence is the last one, 
%\item [ inside] (\textit{I}) a segment in any other cases.
%\end{description}


%in terms of relevance or role in the discourse organisation can be expressed on the new content, on the quoted text or even on the text which is not reused in the reply message.

%In this paper, we choose to assume that the sentences of the quoted text in a reply message can inform

%The original message consists in an homogeneous discourse flow of utterances. 

%By replying to a message and by extracting deliberately some parts\footnote{Summarization operations are also possible.}, the participant performs some cognitive operations leading to identify sufficient information to describe a context.

% So the quoted text in a reply message is assumed to be sufficient. 

% FIXME develop the idea

% The objective is so to determine which parts of the reply messages are reused from the source message.



%\begin{multicols}{2}

%\end{multicols}
%
%As a matter of fact, 
% [STARTEND] if the sentence of the source message is surrounded by insertions which are part of the reply message;
%[START] else if the sentence of the source message is surrounded by insertions which are part of the reply message;
%Par exemple, on pourra étiqueter de \texttt{TERMINE} la phrase précédent un segment repris et de \texttt{DEBUTE} la première phrase du segment repris. Ces phrases ainsi annotées dans leur contexte constitueront notre corpus d'entraînement.
%
Figure~\ref{fig:exampleSegmentationLabels} illustrates the scheme by showing how sentences from Figure~\ref{fig:exampleSourCeReplyMessage} can be aligned and the labels inferred from it. 
%
%This scheme 
It is similar to the \textit{BIO} %annotation 
scheme except it is not at the token level but at the sentence level \cite{ratinov:2009:conll}. % and not at the token level.



%------------------------------------------------------------------------------
\subsection{Annotation generation procedure}
\label{}


Before being able to predict labels of the original message sentences, it is necessary to identify those that are re-used in a reply message. 
Identification of the quoted lines in a reply message is not sufficient for various reasons. 
%
First, the segmenter is intended to work on non-noisy data (i.e. the new content parts in the messages) while a quoted message is an altered version of the original one. 
%Indeed the original message and the future quoted text suffer from multiple transformations. 
% As a matter of fact, messages in a discussion are handled by several distinct email software clients which are not always standards-compliant and totally compatible.
%The main problem comes from the ability to handle non ASCII characters.
%One of these alterations results from encoding and decoding problems of the message content due to the fact that 
Indeed, some email software clients involved in the discussion %handle the messages in a discussion 
are not always standards-compliant and totally compatible\footnote{The \textit{Request for Comments} (RFC) are guidelines and protocols proposed by working groups involved in the Internet Standardization \url{https://tools.ietf.org/html}, the message contents suffer from encoding and decoding problems. Some of the RFC are dedicated to email format and encoding specifications (See RFC 2822 and 5335 as starting points). %The last 20 years, t
There have been several propositions with updates and consequently obsoleted versions which may explain some alteration issues.}. 
In particular, the quoted parts can be wrongly re-encoded at each exchange step due to the absence of dedicated header information. % about it.
%So each of them can 
In addition, the client programs can integrate their own mechanisms for quoting the previous messages when including them as well as for wrapping too long lines\footnote{
%Feature for limiting the line length by splitting them into multiple pieces of no more than 80 characters and make the text readable without any horizontal scrolling.
Feature for making the text readable without any horizontal scrolling by splitting lines into pieces of about 80 characters.}.
%
% Futhermore, because of some user preferences or systems configurations, a thread may embed and interleave various posting styles.  
%
% 
%So, building a model on an altered version of the original message would create a bias in the approach.
%
Second, accessing the original message may allow taking some contextual features into consideration (like the visual layout for example). 
%
Third, to go further, the original context of the extracted text %in the original message
 also conveys some segmentation information. For instance, a sentence from the original message, not present in the reply, but following an aligned sentence, can be considered as starting a segment.

So in addition to identifying the quoted lines, we deploy an alignment procedure to get the original version of the quoted text. 
In this paper, we do not consider the contextual features from the original message and focus only on sentences that have been aligned. 

%While the form could be acceptable for some basic experiment, we decide to deploy an alignment procedure for various reasons: get the original form of the text which is quoted. 











%\begin{figure}
%\centering
%\begin{minipage}{.5\textwidth}

%\begin{description}
%\item [ starting and ending] (\textit{SE}) a segment when there is only one sentence in the segment, 
%\item [ starting] (\textit{S}) a segment if there are at least two sentences in the segment and the sentence is the first one, 
%\item [ ending] (\textit{E}) a segment if there are at least two sentences in the segment and the sentence is the last one, 
%\item [ inside] (\textit{I}) a segment in any other cases.
%\end{description}
%  \captionof{figure}{A figure}
%  \label{fig:test1}
%\end{minipage}%
%\begin{minipage}{.5\textwidth}

%%\begin{figure}[h!]
%\small\centering
%\begin{tabular}{*{2}{|l}|c|}
%\toprule
%\textbf{Source} & \textbf{Reply} & \textbf{Label}\\
%	\midrule
%S1  & & \\
%    & R1 & \\
%\textit{S2}  & > \textit{R2}& \texttt{Start}\\
%\textit{S3}  & > \textit{R3}& \texttt{Inside}\\
%\textit{S4}  & > \textit{R4}& \texttt{End}\\
%    & R5 & \\
%    & R6 & \\
%    & R7 & \\
%    & R8 & \\
%\textit{S5}  & > \textit{R9} & \texttt{Start\&End}\\
%    & R10 & \\
%%    & R11 & \\
%    & [...] & \\
%S7    &  & \\ \ 
%[...] \    &  & \\
%	\bottomrule
%\end{tabular}

%  \captionof{figure}{Sentences alignment of the source and the reply messages from the Figure~\ref{fig:exampleSourCeReplyMessage}. Examples of segmentation labels which can be inferred from the text re-use of the source message and associated to the source sentences.}
%\label{fig:exampleSegmentationLabels}
%\end{minipage}
%\end{figure}



The generation procedure is intended to "automatically" annotate sentences from the original messages with segmentation information.
%Figure~\ref{fig:procedureTrainingDataGeneration} describes the steps of the generation procedure at the global level. 
The procedure follows the following steps:
%\begin{figure}[ht!]
%%\begin{enumerate}
%%\item List the source and the reply messages
%%\item 
%\fbox {
%    \parbox{\linewidth}{
\begin{enumerate}[itemsep=0mm]
%\itemsep0em 
\item Messages posted in the interleaved replying style are identified
\item For each pair of original and reply messages:
\begin{enumerate}
\item Both messages are tokenized at sentence and at word levels
\item Quoted lines in the reply message are identified
\item Sentences which are part of the quoted text in the reply message are identified 
\item Sentences in the original message are aligned with  quoted text in the reply message \footnote{Section~\ref{secalignmentmodule} details how alignment is performed.}
%\item For each aligned sentence 
%\begin{enumerate}
\item Aligned original sentences are labelled in terms of position in segment %(See~Figure~\ref{fig:algoLabellingEachAlignedSentence})
%\end{enumerate}
\item The sequence of labelled sentences is added to the training data %The sequence of labelled sentences makes up a labelled message which is add to the training data 
\end{enumerate}
\end{enumerate}
%}}
%\caption{Procedure for generating the training data}
%\label{fig:procedureTrainingDataGeneration}
%\end{figure}

% Each message from our corpus are tokenized and the sentences split. 
%
Messages with \textit{inline replying} are recognized thanks to the presence of at least two consecutive quoted lines separated by new content lines. 
Pairs of original and reply messages are constituted based on the \texttt{\footnotesize in-reply-to} field present in the email headers.
%The \textit{quoted lines} %in the reply messages 
% are identified based on the presence of a specific prefix at the beginning of the lines. 
 As declared in the RFC~3676\footnote{\url{http://www.ietf.org/rfc/rfc3676.txt}}, we consider as \textit{quoted lines}, the lines %of a message
 beginning with %to be quoted if the first character is 
 the "\texttt{>}" (greater than) sign.
Lines which are not quoted lines are considered to be \textit{new content} lines.
The word tokens are used to index the quoted lines and the sentences. 


Labelling of aligned sentence (sentence from the original message re-used in the reply message) uses this simple rule-based algorithm:
%Figure~\ref{fig:algoLabellingEachAlignedSentence} gives some precisions about the algorithm for labelling each aligned sentence with a segmentation instruction. 
%\begin{lstlisting}
%For each aligned source sentence (sentence from the source message re-used in the reply message)
%  if the sentence is surrounded by new content in the reply message, then label it Start&End
%  else if the sentence is preceded by a new content, then label it Start
%    else if the sentence is followed by a new content, then label it End
%      else label it Inside
%\end{lstlisting}

%\begin{figure}[ht!]
%%\fbox{
%%\begin{enumerate}
%%\item 
%\fbox {
%    \parbox{\linewidth}{
\begin{itemize}
\item[] For each aligned original sentence: \vspace{-0.2cm}
\begin{itemize}
\item[] if the sentence is surrounded by new content in the reply message, the label is \texttt{\footnotesize Start\&End}
\item[] else if the sentence is preceded by a new content, the label is \texttt{\footnotesize Start}
\item[] \indent else if the sentence is followed by a new content, the label is \texttt{\footnotesize End}
\item[] else, the label is \texttt{\footnotesize Inside}
\end{itemize}
\end{itemize}
%    }
%}

%%}
%%\end{enumerate}
%\caption{Algorithm for labelling each aligned sentence with a segmentation instruction}
%\label{fig:algoLabellingEachAlignedSentence}
%\end{figure}
% We assume there is no deletion of original text between two consecutive quoted parts. 



%------------------------------------------------------------------------------
\subsection{Alignment module}
\label{secalignmentmodule}



For finding alignments between two given text messages, we use 
%an implementation of the % Implements a portion of the
% NIST align/scoring 
%algorithm to compare a reference string to a hypothesis string. 
a \textit{dynamic programming (DP) string alignment algorithm} \cite{sankoff:1983}. 
In the context of speech recognition, the algorithm is also known as the \textit{NIST align/scoring algorithm}. Indeed, it is widely used to evaluate the output of speech recognition systems by comparing the hypothesized text %(HYP) 
output by the speech recognizer to the correct, or reference % (REF) 
text. 
%In particular, it is used to compute the word error rate (WER) and the sentence error rate (SER).
%
The %``DP string alignment 
algorithm works by ``performing a global minimization of a Levenshtein distance function which weights the cost of correct words, insertions, deletions and substitutions as 0, 75, 75 and 100 respectively.
%
The computational complexity of DP is $O(MN)$.''
%    final static int MAX_PENALTY = 1000000;
%    final static int SUBSTITUTION_PENALTY = 100;
%    final static int INSERTION_PENALTY = 75;
%    final static int DELETION_PENALTY = 75;
%The alignment and metrics are intended to be, by default, identical to those of the \url{http://www.icsi.berkeley.edu/Speech/docs/sctk-1.2/sclite.htm} NIST SCLITE tool.  
%The program sclite is a tool for scoring and evaluating the output of speech recognition systems. Sclite is part of the NIST SCTK Scoring Tookit. The program compares the hypothesized text (HYP) output by the speech recognizer to the correct, or reference (REF) text. After comparing REF to HYP, (a process called alignment), statistics are gathered during the scoring process and a variety of reports can be produced to summarize the performance of the recognition system.
%\url{http://www1.icsi.berkeley.edu/Speech/docs/sctk-1.2/sclite.htm}



%CMU Sphinx
%Open Source Toolkit For Speech Recognition
%Project by Carnegie Mellon University
% of the performance of a speech recognition or machine translation system.



The Carnegie Mellon University provides an implementation of the algorithm in its speech recognition toolkit\footnote{Sphinx 4 \texttt{edu.cmu.sphinx.util.NISTAlign} %source code.
\url{http://cmusphinx.sourceforge.net}}.
%Implements a portion of the NIST align/scoring algorithm to compare a reference string to a hypothesis string.  It only keeps track of substitutions, insertions, and deletions.
We use an adaptation of it which allows working on lists of strings\footnote{\url{https://github.com/romanows/WordSequenceAligner}} rather than directly on strings (as sequences of characters).


%and other statistics available from an alignment of a hypothesis string and a reference string.










% -----------------------------------------------------------------------------
\section{Building the segmenter}
\label{sec:buildingTheSegmenter}
Each email is processed as a sequence of sentences.
We choose to define the segmentation problem as a sequence labelling task whose aim is to assign the globally best set of labels for the entire sequence at once. The underlying idea is that the choice of the optimal label for a given sentence is dependent on the choices of nearby sentences.
% If a sentence is labelled \texttt{\footnotesize ending}, the next one is likely to be a \texttt{\footnotesize starting}.
%
% wikipedia is a type of pattern recognition task that involves the algorithmic assignment of a categorical label to each member of a sequence of observed values. A common example of a sequence labeling task is part of speech tagging, which seeks to assign a part of speech to each word in an input sentence or document. Sequence labeling can be treated as a set of independent classification tasks, one per member of the sequence. However, accuracy is generally improved by making the optimal label for a given element dependent on the choices of nearby elements, using special algorithms to choose the globally best set of labels for the entire sequence at once.
%
Our email segmenter is built around a linear-chain Conditional Random Field (CRF), as implemented in the sequence labelling toolkit Wapiti \cite{lavergne2010practical}.

Training the classifier to recognize the different labels of the previously defined annotation scheme can be problematic. It has indeed some disadvantages that can undermine the effectiveness of the classifier. In particular, sentences annotated \textit{SE} will, by definition, share important characteristics with sentences bearing the annotation \textit{S} and \textit{E}. 
%For the purposes of segmentation, 
So we chose to transform these annotations into a binary scheme and merely differentiate sentences that starts a new segment (\textit{True}), or "boundary sentences", from those that do not (\textit{False}). The conversion process is trivial, and can easily be reversed\footnote{Sentences labelled with \textit{SE} or \textit{S} are turned into \textit{True}, the other ones into \textit{False}. To reverse the process, a \textit{True} is turned into \textit{SE} if the next sentence is also a boundary (i.e. a True) and into \textit{S} otherwise. While a \textit{False} is turned into \textit{E} if the next sentence is a boundary (i.e. a True) and into  \textit{I} otherwise.}.

We distinguish four sets of features:  $n$-gram features, information structure based features, thematic features and miscellaneous features.
%The purpose of each one of these features is to describe the current sentence from which it is extracted. 
All the features are domain-independant. Almost all features are language-independant as well, save for a few that can be easily translated.
For our experiments, the CRF window size is set at 5, i.e. the classification algorithm takes into account features of the next and previous two sentences as well as the current one.



% -----------------------------------------------------------------------------
\paragraph{$n$-gram features}
%
We select the case-insensitive word bigrams and trigrams with the highest document frequency in the training data (empirically we select the top 1,000 $n$-grams), and check for their presence in each sentence. Since the probability of having multiple occurrences of the same $n$-gram in one sentence are extremely low, we do not record the number of occurrences but merely a boolean value. %The check is case-insensitive.


% -----------------------------------------------------------------------------
\paragraph{Information structure based features}

This feature set is inspired by the information structure theory \cite{kruijff:1996} which 
describes the information imparted by the sentence in terms of the way it is related to prior context. %: topic, (back)ground, focus, new information\ldots
 The theory relates these functions with particular syntactic constructions (e.g. topicalization) and word order constraints in the sentence.
%
%For languages such as English or French, the beginning of the sentence is an important position to structure the information at the discourse level while the ending of the sentence may carry information to announce what the following will deal with.
%The beginning of a sentence is so often seen as playing an important role in discourse structure 

We focus on the first and last three \textit{significant} tokens in the sentence. %, which often serve as connectors to adjoining sentences and therefore hold important structural information.
A token is considered as significant if its occurrence frequency is higher than  1/2,000\footnote{This value was set up empirically on our data. More experimentation needs to be done to generalize it.}.
%Token significance is determined by the number of occurrences of the token in the training corpus: terms with a frequency lower than 1/2000 are considered insignificant. If a sentence contains less than six significant tokens, the same token can be found in both triplets. If the sentence contains less than three significant tokens, missing values are replaced by a placeholder.
As features we use $n$-grams of the surface form, lemma and part-of-speech tag of each triplet (36 features).
%For example, from the sentence "\textit{Have you tried to update your browser?}", we can generate the following features~: \texttt{\footnotesize "Have", "have", "NNP", "you", "you", "PRP", "tried", "try", "VBD", "Have you", "have you", "you tried", "you try", "NNP PRP", "PRP VBD", "Have you tried", "have you try", "NNP PRP VBD", "your", "your", "PRP"} etc.
%
%We define three individual features for each of the three unigrams, the two bigrams and the single trigram found in each of these triplets. The features are the following: the unaltered form of each token (case-sensitive), their lemmatized form (case-insensitive) and their corresponding part-of-speech. This is illustrated in figure ~\ref{fig:exampleSyntacticFeatures}.
%

%\begin{figure}\small\centering
%\begin{tabular}{*{2}{|l}|l|}
%\toprule
%\textbf{Surface form} & \textbf{Lemma} & \textbf{P.O.S.}\\
%	\midrule
%	Many & many & JJ\\
%	thanks & thanks & NNS\\
%	to & to & TO\\
%	your & your & PRP\\
%	suggestions & suggestion & DD\\
%	. & . & .\\
%	Many thanks & many thanks & JJ NNS\\
%	thanks to . & thanks to . & NNS TO . \\
%	your suggestions & your suggestion & PRP DD\\
%	suggestions & suggestion & DD\\
%	Many thanks to & many thanks to & JJ NNS TO\\
%	your suggestions . & your suggestion . & PRP DD .\\
%	\bottomrule
%\end{tabular}

%\caption{Syntactic features formed from the sentence "\textit{Many thanks to all of you for the help you have offered, I have learned tremendously from all your suggestions.}" Each cell is a feature.}
%\label{fig:exampleSyntacticFeatures}
%\end{figure}

% -----------------------------------------------------------------------------
\paragraph{Thematic feature}
%
The only feature we use to account for thematic shift recognition is the output of the TextTiling algorithm \cite{hearst1997texttiling}. TextTiling is one of the most commonly used algorithms for automatic text segmentation. 
%It is a statistical method for thematic segmentation, i.e. a method that attempts to segment texts according to the contrasting themes of its parts: 
If the algorithm detects a rupture in the lexical cohesion %thematic continuity 
of the text (between two consecutive blocks), it will place a boundary to indicate a thematic change. %This method is based on the notion of lexical cohesion.
%
Due to the short size of the messages, we define a block size to equate the sum of three times the sentence average size in our corpus. We set the step-size (overlap size of the rolling window) to the average size of a sentence.

% -----------------------------------------------------------------------------
\paragraph{Miscellaneous features}
%
This feature set includes stylistic %graphic, orthographic 
and semantic features. 24 features, several of them borrowed from related work in speech act classification \cite{qadir2011classifying} and email segmentation \cite{lampert2009segmenting}, are in the set:
%
\textit{Stylistic features} capture information about the visual structure and composition of the message:
the position of the sentence in the email, 
the average length of a token,
the total number of tokens and characters, 
the proportion of upper-case, alphabetic and numeric characters,
%\textit{Orthographic features} capture information about the use of distinctive characters or character sequences:
the number of greater-than signs (``>''); %also know as ``chevron'' symbols, in the sentence
whether the sentence ends with or contains a question mark, a colon or a semicolon; %, or if it at least contains one (three features)
whether the sentence contains any punctuation within the first three tokens (this is meant to recognize greetings \cite{qadir2011classifying}).
%\end{itemize}

\textit{Semantic features} check for meaningful words and phrases:
%
%\begin{itemize}
%	\item 
whether the sentence begins with or contains a ``wh*'' question word 
% (\textit{``who''}, \textit{``when''}, \textit{``where''}, \textit{``what''}, \textit{``which''}, \textit{``what''}, \textit{``how''}) 
or a phrase %an $n$-gram 
suggesting an incoming interrogation (e.g. \textit{``is it''}, \textit{``are there''}); %, and whether the sentence merely contains such a word or $n$-gram
%	\item 
whether the sentence contains a modal; 
%(\textit{``can''}, \textit{``may''}, \textit{``must''}, \textit{``shall''}, \textit{``will''}, \textit{``might''}, \textit{``should''}, \textit{``would''}, \textit{``could''}, and their negative forms);
%	\item 
whether any plan phrases (e.g. \textit{``i will''}, \textit{``we are going to''}) are present;
%	\item 
whether the sentence contains first person (e.g. \textit{``we''}, \textit{``my''}) % and \textit{``me''}), 
second person or third person words; % (e.g. \textit{``we''}, \textit{``my''} and \textit{``me''} are recognized as first person words)
%	\item 
the first personal pronoun found in the sentence;
%	\item 
the first verbal form found.% in the sentence as classified by the Stanford Part-Of-Speech tagger ; that is an element of the Penn Treebank tag set \footnote{Alphabetical list of part-of-speech tags used in the Penn Treebank Project: \url{http://www.ling.upenn.edu/courses/Fall_2003/ling001/penn_treebank_pos.html}} (e.g. the feature \textit{``VBZ''} indicates a present tense verb in third person singular)

%\begin{itemize}
%    \item \textit{SE} $\longrightarrow$ \textit{True} $\longrightarrow$ \textit{SE} (if the next sentence is also a boundary)
%    \item \textit{S} $\longrightarrow$ \textit{True} $\longrightarrow$ \textit{S} (if the next sentence is not a boundary)
%    \item \textit{I} $\longrightarrow$ \textit{False} $\longrightarrow$ \textit{I} (if the next sentence is not a boundary either)
%    \item \textit{E} $\longrightarrow$ \textit{False} $\longrightarrow$ \textit{E} (if the next sentence is a boundary)
%\end{itemize}


% -----------------------------------------------------------------------------
% METHODOLOGY
% -----------------------------------------------------------------------------
\section{Experimental framework}
\label{sec:experimentalframework}
%In this section, 
We describe the data, the preprocessing and the evaluation protocol we use for our experiments.

% -----------------------------------------------------------------------------
\subsection{Corpus}

The current work takes place in a project dealing with multilingual and multimodal discussion processing, mainly in interrogative technical domains. 
For these reasons we did not consider the Enron Corpus (30,000 threads) \cite{klimt:2004:enron} (which is from a corporate environment), 
%30,000 threads
neither the W3C Corpus (despite its technical consistence) or its subset, the British Columbia Conversation Corpus (BC3) \cite{ulrich:2008:bc3}.
%50,000 threads

We rather use the \textit{ubuntu-users} email archive\footnote{Ubuntu mailing lists archives (See \textit{ubuntu-users}): \url{https://lists.ubuntu.com/archives/}} as our primary corpus. It offers a number of advantages. It is free, and distributed under an unrestrictive license. It increases continuously, and therefore is representative of modern emailing in both content and formatting. Additionally, many alternatives archives are available, in a number of different languages, including some very resource-poor languages. Ubuntu also offers a forum and a FAQ which are interesting in the context of multimodal studies. 

%For this work, 
We use a copy of December 2013.
The corpus contains a total of 272,380 messages (47,044 threads). 33,915 of them are posted in the inline replying style that we are interested in. These messages are made of 418,858 sentences, themselves constituted of 76,326 unique tokens (5,139,123 total). 87,950 of these lines (21\%) are automatically labelled by our system as the start of a new segment (either \textit{SE} or \textit{S}).
% -----------------------------------------------------------------------------
\subsection{Evaluation protocol}

In order to evaluate the efficiency of the segmenter, we perform a 10-fold cross-validation on the Ubuntu corpus, and compare its performance to two different baselines. The first one, the ``regular'' baseline, is computed by segmenting the test set into regular segments of the same length as the average training set segment length, rounded up. The second one is the TextTiling algorithm we described in section~\ref{sec:buildingTheSegmenter}.
While it is used as a feature in the proposed approach in the previous section, the direct output of the TextTiling algorithm is used for the baseline.


The results are measured with a panel of metrics used in text segmentation and Information Retrieval (IR).
% -----------------------------------------------------------------------------
%\subsubsection{Metrics}
%
%Traditional
%Information retrieval metrics of 
Precision ($P$) and Recall ($R$) are provided for all results. $P$ is the percentage of boundaries identified by the classifier that are indeed true boundaries. $R$ is the percentage of true boundaries that are identified by the classifier. 
% We also provide the $F_1$ score which represents the harmonic mean of precision and recall:
We also provide the harmonic mean of precision and recall:
%\[
$  F_1 = 2 \cdot \frac{P \cdot R}{P + R}$
%\]

However, automatic evaluation of speech segmentation through these metrics is problematic as predicted segment boundaries seldom align precisely. 
%Moreover, while a segmenter that places boundaries near actual boundary positions is in almost all cases more suited to the task than one that misses by a much larger margin, precision and recall metrics would penalize both to the same extent. 
Therefore, %in order to evaluate varying degrees of success or failure in a more subtle manner,
 we also provide an array of metrics relevant to the field of text segmentation : ${P_{k}}$, \textit{WindowDiff} and the \textit{Generalized Hamming Distance (GHD)}.
%$\bm{P_{k}}$,
%
The ${P_{k}}$ metric is a probabilistically motivated error metric for the assessment of segmentation algorithms \cite{beeferman1999statistical}.
%
\textit{WindowDiff} compares the number of segment boundaries found within a fixed-sized window to the number of boundaries found in the same window of text for the reference segmentation \cite{pevzner2002critique}.
%
The \textit{GHD} is an extension of the Hamming distance\footnote{Wikipedia article on the Hamming distance: \url{http://en.wikipedia.org/wiki/Hamming_distance}} that gives partial credit for near misses \cite{bookstein2002generalized}.
% -----------------------------------------------------------------------------
%\subsubsection{Baselines}


% -----------------------------------------------------------------------------
\subsection{Preprocessing}

To reduce noise in the corpus we filter out undesirable emails based on several criteria, the first of which is encoding. Messages that are not UTF-8 encoded are removed from the selection. The second criterion is MIME type: we keep single-part plain text messages only, and remove those with HTML or other special contents.
%
In addition, we choose to consider only replies to thread starters. This choice is based on the assumption that the alignment module would have more difficulty in recognizing properly sentences that were repeatedly transformed in successive replies. Indeed, these replies - that would contain quoted text from other messages - would be more likely to be poorly labelled through automatic annotation.
%
The last criterion % for message selection
 is length. The dataset being built from a mailing list that can cover very technical discussions, users sometimes send very lengthy messages containing many lines of copied-and-pasted code, software logs, bash command outputs, etc. The number of these messages is marginal, but their lengths being disproportionately high, they can have a negative impact on the segmenter's performance. We therefore exclude messages longer than the average message length plus the standard length deviation.
%
After filtering, the dataset is left with 6,821 messages out of 33,915 (20\%).

For building the segmenter features, we use the Stanford %Log-linear
 Part-Of-Speech Tagger for morpho-syntactic tagging \cite{toutanova2003feature}, and the WordNet lexical database for lemmatization \cite{miller1995wordnet}.



% -----------------------------------------------------------------------------
% RESULT

\newcolumntype{C}[1]{>{\centering\let\newline\\\arraybackslash\hspace{0pt}}m{#1}}

\section{Experiments}
\label{sec:experiments}

% FIXME Table \ref{fig:results} présente / résume / fusionne tous les résultats ; phrase qui annonce les différentes étapes



Table~\ref{fig:results} shows the summary of all obtained results. On the left side are shown results about segmentation metrics, on the right side results about information retrieval metrics. First, we examine baseline scores, and display them in the top section. Second, in the middle section, we show results for segmenters based on individual feature sets (with $A$ standing for $n$-grams, $B$ for information structure, $C$ for TextTiling and $D$ for miscellaneous features). Finally, in the lower section, we show results based on feature sets combinations.


\begin{table}[h]
	\begin{tabular}{p{2.5cm}|C{1.25cm}|C{1.25cm}|C{1.25cm}||C{1.25cm}|C{1.25cm}|C{1.25cm}|}
		\cline{2-7}
		& \multicolumn{3}{c||}{Segmentation metrics} & \multicolumn{3}{c|}{Information Retrieval metrics} \\ \cline{2-7}
		& $WD$ & $P_{k}$ & $GHD$ & $P$ & $R$ & $F_1$ \\ \hline
		\multicolumn{1}{|l|}{regular baseline} 					& .59 & .25 & .60 & .31 & .49 & .38  \\ \hline
		\multicolumn{1}{|l|}{TextTiling baseline} 				& .41 & .07 & .38 & .75 & .44 & .56 \\ \hline\hline
		\multicolumn{1}{|l|}{$\phi(A)$ with $A$ = $n$-grams} 						& .38 & \textbf{.05} & .39 & \textbf{1} & .39 & .56 \\ \hline
		\multicolumn{1}{|l|}{$\phi(B)$ with $B$ = info. structure} 			& .43 & .11 & .38 & .60 & .68 & \textbf{.64} \\ \hline
		\multicolumn{1}{|l|}{$\phi(C)$ with $C$ = TextTiling}				& .39 & .05 & .38 & .94 & .40 & .56 \\ \hline
		\multicolumn{1}{|l|}{$\phi(D)$ with $D$ = misc. features} 					& .41 & .09 & .38 & .69 & .49 & .57 \\ \hline\hline
		\multicolumn{1}{|l|}{$\phi(A + B + C + D)$} 						& .38 & \textbf{.05} & .39 & \textbf{1} & .39 & .56\\ \hline
		\multicolumn{1}{|l|}{$\phi(\phi(A) + \phi(B) + \phi(C) + \phi(D))$} 	& .38 & .06 & .36 & .81 & .47 & .59 \\ \hline
		\multicolumn{1}{|l|}{$\phi(A) \cup \phi(B + C + D)$} 	& .45 & .12 & .40 & .58 & \textbf{.69} & .63 \\ \hline
		\multicolumn{1}{|l|}{$\phi(A) \cup \delta(\phi(B + C + D))$} 	& \textbf{.36} & .06 & \textbf{.34} & .80 & .53 & \textbf{.64} \\ \hline
	\end{tabular}
	\caption{Comparative results between baselines and tested segmenters. All displayed results show \textit{WindowDiff} (\textit{WD}), $P_{k}$ and \textit{GHD} as error rates, therefore a lower score is desirable for these metrics. This contrasts with the three IR scores, for which a low value denotes poor performance. Best scores are shown bolded.}
	\label{fig:results}
\end{table}

\subsection{Baseline segmenters}

The first section of Table \ref{fig:results} shows the results obtained by both of our baselines. Unsurprisingly, TextTiling performs much better than the basic regular segmentation algorithm across all metrics save recall.

\subsection{Segmenters based on individual feature sets}

The second section of Table \ref{fig:results} shows the results for four different classifiers, each trained with a distinct subset of the feature set. The $\phi$ function is the classification function, its parameters are features, and its output a prediction. While all classifiers easily beat the regular baseline, and match the TextTiling baseline when it comes to IR metrics, only the thematic and the $n$-grams segmenters manage to surpass TextTiling when performance is measured by segmentation metrics. In terms of IR scores, the $n$-grams classifier in particular stands out as it manages to achieve an outstanding 100\% precision, although this result is mitigated by a meager 39\% recall. It is also interesting to see that the thematic classifier, based only on contextual information about TextTiling output, performs better than the TextTiling baseline.


\subsection{Segmenters based on feature sets combinations}

The last section of Table \ref{fig:results} shows the results of four different segmenters. The first one, $\phi(A + B + C + D)$, is a simple classifier that takes all available features into account. Its results are exactly identical to that of the $n$-grams classifier, most certainly due to the fact that other features are filtered out due to the sheer number of lexical features. The second one, $\phi(\phi(A) + \phi(B) + \phi(C) + \phi(D))$, uses as features the outputs of the four classifiers trained on each individual feature set. Results show this approach isn't significantly better. The third one, $\phi(A) \cup \phi(B + C + D)$, segments according to the union of the boundaries detected by a classifier trained on $n$-grams features and those identified by a classifier trained on all other features. This idea is motivated by the fact that we know all boundaries found by the $n$-grams classifier to be accurate ($P=1$). Doing this allows the segmenter to obtain the best possible recall ($R=.69)$, but at the expense of precision ($P=.58$). The last one, $\phi(A) \cup \delta(\phi(B + C + D))$, attempts to increase the $n$-grams classifier's recall without sacrificing too much precision by being more selective about boundaries. The $\delta$ function is the "cherry picking" function, which filters out boundaries predicted without sufficient confidence. Only those identified by the $n$-grams classifier and those classified as boundaries with a confidence score of at least .99 by a classifier trained on the other feature sets are considered. This system outperforms all others both in terms of segmentation scores and $F_1$, however it is still relatively conservative and the segmentation ratio (the number of true boundaries divided by the number of guessed boundaries) remains significantly lower than expected, at~0.67. Tuning the minimum confidence score ($c$) allows to adjust $P$ from .58 ($c$ = 0) to 1 ($c$ = 1) and $R$ from .39 ($c$ = 1) to .69 ($c$ = 0).

% -----------------------------------------------------------------------------
% RELATED WORK
%------------------------------------------------------------------------------
\section{Related work}
\label{sec:relatedWork}

Three research areas are directly related to our study:
a) collaborative approaches for acquiring
annotated corpora, b) detection of email structure, and c) sentence alignment.
%
%We can set our approach in the trend of the collaborative approaches for acquiring annotated corpora such as the Game With A Purpose (GWAP) \cite{ahn:2006:computer} or the paid-for crowdsourcing \cite{fort:2011:cl}.
In the \cite{wang:2013:lre}'s taxonomy of the collaborative approaches for acquiring annotated corpora, our approach could be related to the \textit{Wisdom of the Crowds} (WotC) genre where motivators are altruism or prestige to collaborate for the building of a public resource.
As a major difference, we did not initiate the annotation process and consequently we did not define annotation guidelines, design tasks or develop tools for annotating which are always problematic questions.
We have just rerouted \textit{a posteriori} the result of an existing task which was performed in a distinct context.
In our case the burning issue is to determine the adequacy of our segmentation task.
Our work is motivated by the need to identify important snippets of information in messages for applications such as being able to determine whether all the aspects of a customer request were fully considered.
We argue that even if it is not always obvious to tag topically or rhetorically a segment, the fact that it was a human who actually segmented the message ensures its quality.
%
% is another major genre for crowdsourcing. WotC deployments allow members of the general public to collaborate to build a public resource, to predict event outcomes or to estimate difficulty to guess quantities. Wikipedia, the most well-known WotC instance, has different motivators that have changed over time. Initially, altruism and indirect benefit were factors: people contributed articles to Wikipedia not only to help others but also to build a resource that would ultimately help themselves. As Wikipedia matured, the prestige of being a regular contributor or editor became a motivator (Suh et al. 2009).
%
We think that our approach can also be used for determining the relevance of the segments, however it has some limits, and we do not know how labelling segments with dialogue acts may help us do so.

Detecting the structure of a thread is a hot topic. 
%
As mentioned in Section~\ref{sec:intro}, very little works have been done on email segmentation. 
We are aware of recent works in linear text segmentation such as \cite{kazantseva:2011} who addresses the problem by modelling the text as a graph of sentences and by performing clustering and/or cut methods. 
%
Due to the size of the messages (and consequently the available lexical material), it is not always possible to exploit this kind of method. However, our results tend to indicate that we should investigate in this direction nonetheless.
%
By detecting sub-units of information within the message, our work may complement the works of \cite{li:2011:threadlinking,kim:2010:taggingandlinking} who propose solutions for detecting links between messages. 
% \texttt{in-reply-to} link
We may extend these approaches by considering the possibility of pointing from/to multiple message sources/targets. % or determining more precisely the pointed area.

Concerning the alignment process, our task can be compared to the detection of monolingual text derivation (otherwise called plagiarism, near–duplication, revision). \cite{poulard:2011:detecting} compare, for instance, the use of $n$–grams overlap with the use of text hapax. 
In constrast, we already know that a text (the reply message) derives from another (the original message). Sentence alignment has also been a very active field of research 
%both in monolingual (e.g. plagiarism detection) and multilingual  (e.g. 
in statistical machine translation for building parallel corpora. %) domains. 
%
%In MT, 
Some methods are based on sentence length comparison \cite{gale:1991}, some methods rely on the overlap of rare words (cognates and named entities) \cite{enright-kondrak:2007:ShortPapers}.
%For detection of derivation links, \cite{poulard:2011:detecting} compare the use of n–grams overlap with the use of text specificities. % exploitation of the specificity and invariance of textual elements. 
% between texts (otherwise called plagiarism, near–duplication, revision, etc.) at the document level. 
%We evaluate the use of textual elements implementing the ideas of specificity and invariance as well as their combination to characterize derivatives. We built a French press corpus based on Wikinews 
% revisions to run this evaluation. We obtain performances similar to the state of the art method 
% (n–grams overlap) while reducing the signature size and so, the processing costs. In order ...
In comparison, %to the speech recognition and translation use cases, 
%our work is more an alignment task than a detection of derivation. In addition 
in our task, despite some noise, the compared text includes large parts of material identical to the original text. 
The kinds of edit operation in presence (no inversion\footnote{When computing the Levenshtein distance, the inversion edit operation is the most costly operation.} only deletion, insertion and substitution) lead us to consider the Levenshtein distance as a serious option.  

% \url{http://www.statmt.org/survey/Topic/SentenceAlignment}
% An influential early method is based on sentence length, measured in words (Brown et al., 1991; Gale and Church, 1991; Gale and Church, 1993) or characters
% training models with parallel texts
%Enright and Kondrak (2007) use a simple and fast method for document alignment that relies of overlap of rare but identically spelled words, which are mostly cognates, names, and numbers.



% -----------------------------------------------------------------------------
% FUTURE WORK


%------------------------------------------------------------------------------
\section{Future work}
\label{sec:futureWork}

The main contribution of this work is to exploit the human effort dedicated to reply formatting for training discursive email segmenters. 
We have implemented and tested various segmenter models. 
There is still room for improvement, but our results indicate that the approach merits more thorough examination.
%
Our segmentation approach remains relatively simple and can be easily extended. One way would be to consider contextual features in order to characterize the sentences in the original message structure.
%
As future works, we plan to complete our current experiments with two new approaches for evaluation. The first one will consists in comparing the automatic segmentation with those performed by human annotators.
This task remains tedious since it will then be necessary to define an annotation protocol, write guidelines and build other resources.
The second evaluation we plan to perform is an extrinsic evaluation. The idea will be to measure the contribution of the segmentation in the process of detecting the dialogue acts, i.e. to check if existing sentence-level classification systems would perform better with such contextual information. % at segment-level.

%TODO
%\begin{itemize}
%\item NH finish the state-of-the-art, find a new example and handle/describe the example and our idea
%\item baseline start at each new sentence
%\item segmentation de choi as feature
%\item feature n-grams : only most frequent unigrams
%\item feature as individual sets: graphic+orthographic~stylistic, semantic
%\item feature contextual: empty lines before/after ; line position, lexical similarity before/after sentence
%\item combination of the individual decision of the segmenters
%\item alternatives of ubuntus allows us to build models for many languages at low cost
%\item alignements et étiquetage sur les mots
%\item évaluation par rapport à annotation manuelle
%\item évaluation de apport par rapport à reconnaissance automatique des DA 
%\item un exemple de segmentation que l'on souhaite
%\item aligner seulement les quoted lines et non tout le contenu du reply
%\item travailler la tokenization des emails
%\item Focus on the first message of a thread and its reply messages to avoid noisy data due to several levels of quoted lines.
%\item Restrict on short messages as an heuristic to filter out the long code trace message
%\item Due to alignment complexity focus only on the first 25 000 tokens.
%\item In source message, focus on part which is new ; in reply message, focus on part which is new + only the first? quoted level 
%\item The classifiers are based on common features for discourse analysis \cite{joty:2013:acl}.
%\item why processing thread is difficult: Futhermore, because of some user preferences or systems configurations, a thread may embed and interleave various posting styles.  ...
 
%\end{itemize}

%\section*{Acknowledgments}

%Do not number the acknowledgment section. Do not include this section when submitting your paper for review.



% include your own bib file like this:
\bibliographystyle{konvens2014}
\bibliography{refs}


\end{document}
