
\section{Approach}
\label{sec:approach}

We present the assumptions and the detailed steps of our approach.

\subsection{Annotation scheme}
\label{sec:annotationscheme}

The basic idea is to interpret the operation performed by a discussion participant on the message he replies to as an annotation operation. 
%
% Assumptions about the kind of annotations depend on the operation that has been performed.
% 
% Deletion or re-use of the original text material can give hints about the relevance of the content: discarded material is probably less relevant than re-used one.
% 
We assume that by replying inside a message and by only including some specific parts, the participant performs some cognitive operations to identify homogeneous self-contained text segments.
% 
Consequently, we make some assumptions about the role played by the sentences in the original message information structure.

A sentence in a segment plays one of the following roles:
%\indent $\bullet$ 
\texttt{\footnotesize starting and ending} (\textit{SE}) a segment when there is only one sentence in the segment, %\\
%\indent $\bullet$ 
\texttt{\footnotesize starting} (\textit{S}) a segment if there are at least two sentences in the segment and it is the first one, %\\
%\indent $\bullet$
\texttt{\footnotesize ending} (\textit{E}) a segment if there are at least two sentences in the segment and it is the last one, %\\
%\indent $\bullet$
\texttt{\footnotesize inside} (\textit{I}) a segment in any other cases.

Figure~\ref{fig:exampleSegmentationLabels} illustrates the scheme by showing how sentences from Figure~\ref{fig:exampleSourceReplyMessage} can be aligned and the labels inferred from it. 
% 
% It is similar to the \textit{BIO} scheme except it is not at the token level but at the sentence level \cite{ratinov:2009:conll}.

\subsection{Annotation generation procedure}
\label{}

Before being able to predict labels of the original message sentences, it is necessary to identify those that are re-used in a reply message. 
%
Identification of the quoted lines in a reply message is not sufficient for various reasons. 

First, the segmenter is intended to work on non-noisy data (i.e. the new content parts in the messages) while a quoted message is an altered version of the original one. 
% 
Indeed, some email software clients involved in the discussion are not always standards-compliant and totally compatible\footnote{The \textit{Request for Comments} (RFC) are guidelines and protocols proposed by working groups involved in the Internet Standardization \url{https://tools.ietf.org/html}, the message contents suffer from encoding and decoding problems. Some of the RFC are dedicated to email format and encoding specifications (See RFC 2822 and 5335 as starting points). There have been several propositions with updates and consequently obsoleted versions which may explain some alteration issues.}. 
%
In particular, the quoted parts can be wrongly re-encoded at each exchange step due to the absence of dedicated header information.
% 
In addition, the client programs can integrate their own mechanisms for quoting the previous messages when including them as well as for wrapping too long lines\footnote{Feature for making the text readable without any horizontal scrolling by splitting lines into pieces of about 80 characters.}.

Second, accessing the original message may allow taking some contextual features into consideration (like the visual layout for example). 

Third, to go further, the original context of the extracted text also conveys some segmentation information. For instance, a sentence from the original message, not present in the reply, but following an aligned sentence, can be considered as starting a segment.

So in addition to identifying the quoted lines, we deploy an alignment procedure to get the original version of the quoted text. 
In this paper, we do not consider the contextual features from the original message and focus only on sentences that have been aligned. 
% 
The generation procedure is intended to "automatically" annotate sentences from the original messages with segmentation information.
% 
The procedure follows the following steps:

\begin{enumerate}[itemsep=0mm]
	\item Messages posted in the interleaved replying style are identified
	\item For each pair of original and reply messages:
	\begin{enumerate}
		\item Both messages are tokenized at sentence and at word levels
		\item Quoted lines in the reply message are identified
		\item Sentences which are part of the quoted text in the reply message are identified 
		\item Sentences in the original message are aligned with  quoted text in the reply message \footnote{Section~\ref{secalignmentmodule} details how alignment is performed.}
		\item Aligned original sentences are labelled in terms of position in segment 
		\item The sequence of labelled sentences is added to the training data 
	\end{enumerate}
\end{enumerate}

Messages with \textit{inline replying} are recognized thanks to the presence of at least two consecutive quoted lines separated by new content lines.
% 
Pairs of original and reply messages are constituted based on the \texttt{\footnotesize in-reply-to} field present in the email headers.
% 
As declared in the RFC~3676\footnote{\url{http://www.ietf.org/rfc/rfc3676.txt}}, we consider as \textit{quoted lines}, the lines beginning with the "\texttt{>}" (greater than) sign.
% 
Lines which are not quoted lines are considered to be \textit{new content} lines.
% 
The word tokens are used to index the quoted lines and the sentences. 
%
The labelling of aligned sentences (sentences from the original message re-used in the reply message) is performed by this simple rule-based algorithm:

\begin{itemize}
	\item[] For each aligned original sentence: \vspace{-0.2cm}
	\begin{itemize}
		\item[] if the sentence is surrounded by new content in the reply message, the label is \texttt{\footnotesize Start\&End}
		\item[] else if the sentence is preceded by a new content, the label is \texttt{\footnotesize Start}
		\item[] \indent else if the sentence is followed by a new content, the label is \texttt{\footnotesize End}
		\item[] else, the label is \texttt{\footnotesize Inside}
	\end{itemize}
\end{itemize}

\subsection{Alignment module}
\label{secalignmentmodule}

For finding alignments between two given text messages, we use a \textit{dynamic programming (DP) string alignment algorithm} \cite{sankoff:1983}.
% 
In the context of speech recognition, the algorithm is also known as the \textit{NIST align/scoring algorithm}. Indeed, it is widely used to evaluate the output of speech recognition systems by comparing the hypothesized text output by the speech recognizer to the correct, or reference text. 
% 
In particular, it is used to compute the word error rate (WER) and the sentence error rate (SER).
% 
The algorithm works by ``performing a global minimization of a Levenshtein distance function which weights the cost of correct words, insertions, deletions and substitutions as 0, 75, 75 and 100 respectively. The computational complexity of DP is $O(MN)$.''
% 
The Carnegie Mellon University provides an implementation of the algorithm in its speech recognition toolkit\footnote{Sphinx 4 \texttt{edu.cmu.sphinx.util.NISTAlign} \url{http://cmusphinx.sourceforge.net}}.
%
We use an adaptation of it which allows working on lists of strings\footnote{\url{https://github.com/romanows/WordSequenceAligner}} rather than directly on strings (as sequences of characters).