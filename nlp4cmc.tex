%
% File konvens2014.tex
%
% This file is identical to the file konvens2012.tex 
% which was provided by  Jeremy Jancsary, <jeremy.jancsary@ofai.at>
% We have renamed it only so it is clear to authors who submit to 
% Konvens 2014 that they are using the right version.
%
% Contact: konvens2014@uni-hildesheim.de
%%
%% Based on the style files for ACL2008 by Joakim Nivre and Noah Smith
%% and that of ACL2010 by Jing-Shin Chang and Philipp Koehn
%% and that of ACL2012 by Maggie Lie and Michael White.

\documentclass[11pt,a4paper]{article}
\usepackage{konvens2014}
\usepackage[pass]{geometry}
\usepackage{times}
\usepackage{latexsym}
\usepackage{amsmath}
\usepackage{multirow}
\usepackage{url}
\usepackage{bm}
\usepackage{todonotes}
\usepackage{csquotes}
\usepackage{float}
\usepackage{array}

% subfigure
\usepackage{graphicx}
\usepackage{caption}
\usepackage{subcaption}

% Package pour les tableaux 
\usepackage{booktabs}

% plusieurs colonnes et modification de leur taille dans les tables
%\usepackage{multirow}

%\usepackage{setspace}
\usepackage{listings}
\lstset{ %
  basicstyle=\footnotesize
}

%\usepackage{rst}
\usepackage{multicol}
\usepackage{enumitem}

\DeclareMathOperator*{\argmax}{arg\,max}
\setlength\titlebox{6.5cm}    % Expanding the titlebox

\title{Exploiting the human computational effort dedicated to message reply formatting for email segmentation}

\date{}

\begin{document}

\maketitle

\begin{abstract}
In the context of multi-domain and multi-modal online asynchronous discussion analysis, we propose an innovative strategy for the segmentation of email messages into meaningful fragments. The originality of the proposed approach lies in its exploitation of the human computational efforts dedicated to message reply formatting to identify text boundaries in email corpora. We describe the approach, propose a new electronic mail corpus and report the comparison of the system to several existing segmenters.
\end{abstract}

% INTRODUCTION

\section{Introduction}
\label{sec:introduction}

Automatic processing of online conversations (forum, emails) is a highly important issue for industrial and scientific communities interested in improving existing question/answering systems, identify emotions or intentions in customer requests or reviews, detect messages containing requests for action, identify unsolved problems etc.

% In most works, conversation interactions between the participants are modeled in terms of dialogue acts (DA) \cite{austin:1970}. 
%
% The DAs describe the communicative function conveyed by each text utterance  (e.g.~question, answer, greeting,\ldots).
%
In this paper, we address the problem of segmenting messages in online asynchronous discussions. 
% The process aims at supporting the analysis of messages in terms of DA.
%
We pay special attention to the processing of electronic mails.

% The main trend in automatic DA recognition consists in using supervised learning algorithms to predict the DA conveyed by a sentence or a message \cite{joty:2013:sigdial}.
%
% The hypothesized message segmentation results from the global analysis of these individual predictions over each sentence.
%
% A first remark on this paradigm is that it is not realistic to use in the context of multi-domain and multimodal processing because it requires the building of training data which is a very substantial and time-consuming task.
%
% A second remark is that the model does not have a fine-grained representation of the message structure or the relations between messages. Considering such characteristics could drastically improve the systems to allow to focus on specific text parts or to filter out less relevant ones.
%
% Indeed, apart from the closing formula, a message may for example be made of several distinct information requests, the description of an unsuccessful procedure, the quote of third-party messages\ldots
%
So far, a few works address the problem of message segmentation.
%
\cite{lampert:2009:emnlp} propose to segment emails in prototypical zones such as the author's contribution, quotes of original messages, the signature, the opening and closing formulas. 
%
In comparison, we focus on the segmentation of the author's contribution (what we call the new content part).
%
\cite{joty:2013:jair} identifies clusters of topically related sentences through the multiple messages of a thread, without distinguishing email and forum messages. Our problem differs because we are only interested in the cohesion between sentences in nearby fragments and not on distant sentences.

% Despite the drawbacks mentioned above, a supervised approach remains the most efficient and reliable method to solve classification problems in Natural Language Processing. 
%
% Our aim is to train a system to detect the segment boundaries, i.e. to determine, through a classification approach, if a given sentence starts, ends or continues a segment.

\begin{figure}
    \begin{minipage}{.46\textwidth}
        \fbox {
            \parbox{\linewidth}{
                \begin{subfigure}[b]{0.99\textwidth}
                    \small
                    [Hi!]$^{S1}$\vspace{0.1cm}

                    [I got my ubuntu cds today and i'm really impressed.]$^{S2}$ [My friends like them and my \\ \
                    teachers too (i'm a student).]$^{S3}$ [It's really funny to see, how people like ubuntu and start feeling geek and blaming microsoft when they use it.]$^{S4}$ \vspace{0.1cm}

                    [Unfortunately everyone wants an ubuntu cd, so can i download the cd covers anywhere or an \\ \
                    'official document' which i can attach to self-burned cds?]$^{S5}$\vspace{0.1cm}

                    [I searched the entire web site but found nothing.]$^{S6}$ [Thanks in advance.]$^{S7}$\vspace{0.1cm}

                    [John]$^{S8}$

                    \caption{Original message.}
                    \label{fig:exampleSource}
                \end{subfigure}%
            }
        }
        \vspace{0.2cm}
        \\
        \fbox {
            \parbox{\linewidth}{
                \begin{subfigure}[b]{0.99\textwidth}
                    \small
                    [On Sun, 04 Dec 2005, John Doe 
                    <john@doe.com> wrote:]$^{R1}$\vspace{0.1cm}

                    \textgreater [I got my ubuntu cds today and i'm really \\ \ 
                    \textgreater impressed.]$^{R2}$ [My friends like them and my \\ \
                    \textgreater teachers too (i'm a student).]$^{R3}$ \\ \ 
                    \textgreater [It's really funny to see, how people like \\ \ 
                    \textgreater ubuntu and start feeling geek and blaming \\ \ 
                    \textgreater microsoft when they use it.]$^{R4}$\vspace{0.1cm}

                    [Rock!]$^{R5}$\vspace{0.1cm}

                    \textgreater [Unfortunately everyone wants an ubuntu cd, so \\ \ 
                    \textgreater can i download the cd covers anywhere or an \\ \ 
                    \textgreater 'official document' which i can attach to self- \\ \ 
                    \textgreater burned cds?]$^{R6}$\vspace{0.1cm}

                    [We don't have any for the warty release, but we will have them for hoary, %\\ \ 
                    because quite a few people have asked. :-)]$^{R7}$\vspace{0.1cm}

                    [Bob.]$^{R8}$ %\vspace{0.1cm}

                    \caption{Reply message.}
                    \label{fig:exampleReply}
                \end{subfigure}
            }
        }

        \caption{An original message and its reply (\textit{ubuntu-users} email archive). Sentences have been tagged to facilitate the discussion.}
        \label{fig:exampleSourceReplyMessage}
    \end{minipage}
    \hfill
    \begin{minipage}{.46\textwidth}
        \small\centering
        \begin{tabular}{*{2}{|l}|c|}
        \toprule
        \textbf{Original} & \textbf{Reply} & \textbf{Label}\\
        	\midrule
            S1  & & \\
            & R1 & \\
            \textit{S2}  & \textgreater \textit{R2}& \texttt{Start}\\
            \textit{S3}  & \textgreater \textit{R3}& \texttt{Inside}\\
            \textit{S4}  & \textgreater \textit{R4}& \texttt{End}\\
            & R5 & \\
            \textit{S5}  & \textgreater \textit{R6} & \texttt{Start\&End}\\
            & R7 & \\
            & [...] & \\
            S6    &  & \\ \ 
            [...] \    &  & \\
        	\bottomrule
        \end{tabular}
        \caption{Alignment of the sentences from the original and reply messages shown in Figure~\ref{fig:exampleSourceReplyMessage} and labels inferred from the re-use of the original message text. Labels are associated to the original sentences.}
        \label{fig:exampleSegmentationLabels}
    \end{minipage}
\end{figure}

% The originality of the proposed approach is to avoid manually annotating the training data and instead to exploit the human computational efforts dedicated to a similar task in a different context of production~\cite{ahn:2006:computer}. 
The originality of the proposed approach is to exploit the human computational efforts dedicated to a similar task in a different context of production~\cite{ahn:2006:computer}. 

As recommended by the \textit{Netiquette}\footnote{Set of guidelines for Network Etiquette (\textit{Netiquette}) when using network communication or information services RFC1855.}, when replying to a message (email or forum post), the writer should ``summarize the original message at the top of its reply, or include (or \"quote\") just enough text of the original to give a context, in order to make sure readers understand when they start to read the response\footnote{It is true that some email software clients do not conform to the recommendations of Netiquette and that some online participants are less sensitive to arguments about posting style (many writers reply above  the original message). We assume that there are enough messages with inline replying available to build our training data.}.''  As a corollary, the writer should ``edit out all the irrelevant material.''
%
Our idea is to use this effort, in particular when the writer replies to a message by inserting his response or comment just after the quoted text appropriate to his intervention. 
%
This posting style is called \textit{interleaved} or \textit{inline replying}.
%
% The so built segmentation model should be usable for any posting styles by applying it only on new content parts.

Figure~\ref{fig:exampleSource} shows an example of an \textit{original} message and, Figure~\ref{fig:exampleReply}, one of its \textit{reply}.
%
We can see that the reply message re-uses only four selected sentences from the original message; namely $S2$, $S3$, $S4$ and $S5$ which respectively correspond to sentences  $R2$, $R3$, $R4$ and $R6$ in the reply message.
%
The author of the reply message deliberately discarded the remaining of the original message.
%
% The segment composed of sentences $S2$, $S3$, $S4$ and the one by the single sentence $S5$ can respectively be associated with two acts : a comment and a question.
We postulate that these informations can be used to hypothesize the presence of boundaries in the original message. 

In Section~\ref{sec:approach}, we explain our approach for identifying boundaries from pairs of source and reply messages.
%
After presenting our experimental framework in Section~\ref{sec:experimental_framework}, we report some evaluations for the segmentation task in Section~\ref{sec:results}.
%
Finally, we discuss our approach in comparison to other works in Section~\ref{sec:related_work}.

% APPROACH

\section{Approach}
\label{sec:approach}

We present the assumptions and the detailed steps of our approach.

\subsection{Annotation scheme}
\label{sec:annotationscheme}

The basic idea is to interpret the operation performed by a discussion participant on the message he replies to as an annotation operation. 
%
% Assumptions about the kind of annotations depend on the operation that has been performed.
% 
% Deletion or re-use of the original text material can give hints about the relevance of the content: discarded material is probably less relevant than re-used one.
% 
We assume that by replying inside a message and by only including some specific parts, the participant performs some cognitive operations to identify homogeneous self-contained text segments.
% 
Consequently, we make some assumptions about the role played by the sentences in the original message information structure.

A sentence in a segment plays one of the following roles:
%\indent $\bullet$ 
\texttt{\footnotesize starting and ending} (\textit{SE}) a segment when there is only one sentence in the segment, %\\
%\indent $\bullet$ 
\texttt{\footnotesize starting} (\textit{S}) a segment if there are at least two sentences in the segment and it is the first one, %\\
%\indent $\bullet$
\texttt{\footnotesize ending} (\textit{E}) a segment if there are at least two sentences in the segment and it is the last one, %\\
%\indent $\bullet$
\texttt{\footnotesize inside} (\textit{I}) a segment in any other cases.

Figure~\ref{fig:exampleSegmentationLabels} illustrates the scheme by showing how sentences from Figure~\ref{fig:exampleSourceReplyMessage} can be aligned and the labels inferred from it. 
% 
% It is similar to the \textit{BIO} scheme except it is not at the token level but at the sentence level \cite{ratinov:2009:conll}.

\subsection{Annotation generation procedure}
\label{}

Before being able to predict labels of the original message sentences, it is necessary to identify those that are re-used in a reply message. 
%
Identification of the quoted lines in a reply message is not sufficient for various reasons. 

First, the segmenter is intended to work on non-noisy data (i.e. the new content parts in the messages) while a quoted message is an altered version of the original one. 
% 
Indeed, some email software clients involved in the discussion are not always standards-compliant and totally compatible\footnote{The \textit{Request for Comments} (RFC) are guidelines and protocols proposed by working groups involved in the Internet Standardization \url{https://tools.ietf.org/html}, the message contents suffer from encoding and decoding problems. Some of the RFC are dedicated to email format and encoding specifications (See RFC 2822 and 5335 as starting points). There have been several propositions with updates and consequently obsoleted versions which may explain some alteration issues.}. 
%
In particular, the quoted parts can be wrongly re-encoded at each exchange step due to the absence of dedicated header information.
% 
In addition, the client programs can integrate their own mechanisms for quoting the previous messages when including them as well as for wrapping too long lines\footnote{Feature for making the text readable without any horizontal scrolling by splitting lines into pieces of about 80 characters.}.

Second, accessing the original message may allow taking some contextual features into consideration (like the visual layout for example). 

Third, to go further, the original context of the extracted text also conveys some segmentation information. For instance, a sentence from the original message, not present in the reply, but following an aligned sentence, can be considered as starting a segment.

So in addition to identifying the quoted lines, we deploy an alignment procedure to get the original version of the quoted text. 
In this paper, we do not consider the contextual features from the original message and focus only on sentences that have been aligned. 
% 
The generation procedure is intended to "automatically" annotate sentences from the original messages with segmentation information.
% 
The procedure follows the following steps:

\begin{enumerate}[itemsep=0mm]
	\item Messages posted in the interleaved replying style are identified
	\item For each pair of original and reply messages:
	\begin{enumerate}
		\item Both messages are tokenized at sentence and at word levels
		\item Quoted lines in the reply message are identified
		\item Sentences which are part of the quoted text in the reply message are identified 
		\item Sentences in the original message are aligned with  quoted text in the reply message \footnote{Section~\ref{secalignmentmodule} details how alignment is performed.}
		\item Aligned original sentences are labelled in terms of position in segment 
		\item The sequence of labelled sentences is added to the training data 
	\end{enumerate}
\end{enumerate}

Messages with \textit{inline replying} are recognized thanks to the presence of at least two consecutive quoted lines separated by new content lines.
% 
Pairs of original and reply messages are constituted based on the \texttt{\footnotesize in-reply-to} field present in the email headers.
% 
As declared in the RFC~3676\footnote{\url{http://www.ietf.org/rfc/rfc3676.txt}}, we consider as \textit{quoted lines}, the lines beginning with the "\texttt{>}" (greater than) sign.
% 
Lines which are not quoted lines are considered to be \textit{new content} lines.
% 
The word tokens are used to index the quoted lines and the sentences. 
%
The labelling of aligned sentences (sentences from the original message re-used in the reply message) is performed by this simple rule-based algorithm:

\begin{itemize}
	\item[] For each aligned original sentence: \vspace{-0.2cm}
	\begin{itemize}
		\item[] if the sentence is surrounded by new content in the reply message, the label is \texttt{\footnotesize Start\&End}
		\item[] else if the sentence is preceded by a new content, the label is \texttt{\footnotesize Start}
		\item[] \indent else if the sentence is followed by a new content, the label is \texttt{\footnotesize End}
		\item[] else, the label is \texttt{\footnotesize Inside}
	\end{itemize}
\end{itemize}

\subsection{Alignment module}
\label{secalignmentmodule}

For finding alignments between two given text messages, we use a \textit{dynamic programming (DP) string alignment algorithm} \cite{sankoff:1983}.
% 
In the context of speech recognition, the algorithm is also known as the \textit{NIST align/scoring algorithm}. Indeed, it is widely used to evaluate the output of speech recognition systems by comparing the hypothesized text output by the speech recognizer to the correct, or reference text. 
% 
In particular, it is used to compute the word error rate (WER) and the sentence error rate (SER).
% 
The algorithm works by ``performing a global minimization of a Levenshtein distance function which weights the cost of correct words, insertions, deletions and substitutions as 0, 75, 75 and 100 respectively. The computational complexity of DP is $O(MN)$.''
% 
The Carnegie Mellon University provides an implementation of the algorithm in its speech recognition toolkit\footnote{Sphinx 4 \texttt{edu.cmu.sphinx.util.NISTAlign} \url{http://cmusphinx.sourceforge.net}}.
%
We use an adaptation of it which allows working on lists of strings\footnote{\url{https://github.com/romanows/WordSequenceAligner}} rather than directly on strings (as sequences of characters).

% METHODOLOGY

\section{Experimental Framework}
\label{sec:experimental_framework}

% RESULT
\input{sections/results.tex}

% RELATED WORK
\input{sections/related_work.tex}

% FUTURE WORK
\input{sections/future_work.tex}
%\section*{Acknowledgments}

%Do not number the acknowledgment section. Do not include this section when submitting your paper for review.

\bibliographystyle{konvens2014}
\bibliography{refs}

\end{document}